The controllability analysis of an ethanol water binary distillation column was conducted. The model for the column was obtained from research done by \textcite{ogun}. The method used for the controllability analysis is outlined by \textcite{skogestad}.

There are some limitations on the system's bandwidth due to open right hand plane poles and delays inherent to the system. It was proven that the bounds set for disturbance rejection and set-point variation were too widely spaced. The system would not be able to reject far reaching disturbances, or track far reaching set-points. An uncertainty model was also developed successfully, and together with the development of a performance weighting function the system can now be tested for robust stability and performance when the controller design takes place.

It is recommended that the bound on upper and lower disturbances to reject, or set-points to track, be re-evaluated. If the bounds are immovable it is recommended that the sizing of the final control elements be re-evaluated.