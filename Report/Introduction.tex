\section{Introduction}

Renewable energy is becoming a major role player in the world today. As people are starting to shift away from fossil fuel based technology and energy generation, the focus is shifting to alternative methods of energy (and fuel) production.

One such a method involves the use of ethanol as an alternative fuel. Ethanol particularly is an excellent contender for a major alternative fuel, as it can be produced from crops or by means of biological fermentation. [A lot of] research is currently done on methods to generate ethanol in order for it to power the future. 

Producing ethanol, however, is not the only problem to overcome. After the production process (usually from fermentation \parencite{henstra}), the ethanol has to be separated from the product mixture in order to purify it. This poses to be a challenge (and a very energy intensive operation) due to the thermodynamic properties of the homogeneous mixture between water and ethanol. As noted in Figure~\ref{fig:vle-ethanol}, the system contains an azeotrope \parencite{wankat}. This leads to very expensive separation operations, as pressure swing distillation has to be implemented for high purity separation \parencite{perry}.

\begin{figure}[tbph!]
	\centering
	\includegraphics[width=0.7\linewidth]{"Figures/VLE Ethanol 2"}
	\caption{Vapour Liquid Equilibrium (VLE) data for the water ethanol system at 1 bar.}
	\label{fig:vle-ethanol}
\end{figure}

In this report, an investigation regarding the controllability of the ethanol water separation process is investigated. The plant investigated is a pilot plant that is testing the feasibility for scale up of the process. Control has to implemented to ensure that the system reaches a steady state, as well as to improve the overall profitability by reducing the standard deviation in the product quality.

A plant wide control system will not be investigated. Only the distillation coloumn is analysed. The current proposed control system is discussed in Section~\ref{sec:Process Model Description}. This report will investigate the validity of such a proposed control scheme, as well as where the physical constrains in the system lies.